% Abstract
\section*{\centering\fontsize{14}{16}\selectfont Abstract}
\addcontentsline{toc}{section}{Abstract}
\vspace{3\baselineskip}
\thispagestyle{plain}
\setstretch{1.5}
{\justifying\fontsize{12}{14}\selectfont
Wave energy converters (WECs) that require reactive power often need more complex power
take-off (PTO) devices, as compared to those that do not require it. WECs that require reactive
power often have more complex power take-off (PTO) devices compared to those that do not
require it. Variable Shape Buoy Wave Energy Converters (VSB WECs) were developed to
convert energy at a higher rate while reducing the need for reactive power. ANSYS's 2-way
fluid-structure interaction (FSI) tool was used in a Computational Fluid Dynamics (CFD) tool to
model VSB WECs. A CFD-based Numerical Wave Tank was developed to simulate open sea
conditions (CNWT) and evaluate the nonlinear response of the VSB WECs. Results show that
the VSB WECs undergo significant deformation before reaching steady-state behavior in
response to approaching waves, which is more accurate than previous low-fidelity dynamic
models. The study concludes that under suitable conditions, the variance in motion can be
utilized to capture energy at a faster rate without requiring power flow.
}
\clearpage